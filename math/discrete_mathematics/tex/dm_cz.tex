\documentclass[12pt, letterpaper, twoside]{article}

% \usepackage[utf8]{inputenc}
% \usepackage{hyperref}
% \hypersetup{
%     colorlinks,
%     citecolor=black,
%     filecolor=black,
%     linkcolor=black,
%     urlcolor=black
% }
\usepackage{amsfonts}
\usepackage{amssymb}
% \usepackage{amsmath}

\title{Diskrétní matematika}
\author{Jaroslav Langer \thanks{z přednášek BI-ZDM/FIT/ČVUT}}
\date{Říjen 2020}

\begin{document}

\maketitle

\tableofcontents

\section{Přednáška 1.}

\section{Přednáška 5.}

\section{Přednáška 10.}

\subsection{Základy teorie čísel}

\subsection{Dělitelnost}

\subsubsection*{Definice 1 (Dělitelnost)}
Nechť $a,b \in \mathbb{Z}$, $a$ dělí $b$, značíme $a \mid b$, pokud $\exists k \in \mathbb{Z}$ takové, že
$b = k \cdot a$.
Říkáme, že $a$ je \textbf{dělitel} $b$, nebo že $b$ je \textbf{násobek} $a$, nebo, že $b$ je \textbf{dělitelné} $a$.
Pokud $a$ nedělí $b$, značíme $a \nmid b$

\subsection{Modulární aritmetika}

\end{document}