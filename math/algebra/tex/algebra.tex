\documentclass[12pt, letterpaper]{article}

% \usepackage[utf8]{inputenc}
% \usepackage{hyperref}
% \hypersetup{
%     colorlinks=true,
%     linkcolor=color,
%     filecolor=magenta,      
%     urlcolor=blue,
% }

\usepackage{amsmath}
\usepackage{amsfonts}

\title{Algebra}
\author{Jaroslav Langer \thanks{z přednášek NI-MPI/FIT/ČVUT}}
\date{Říjen 2020}



\begin{document}

\maketitle

\tableofcontents

\begin{abstract}
Definice, věty a poznámky z předmětu NI-MPI.
% \href{https://courses.fit.cvut.cz/NI-MPI/lectures/index.html}{Odkaz na courses}.
\end{abstract}

\section{Přednáška 1.}

\subsection{Úvod}

\subsubsection*{Věta 1.1}
Pro všechna $b,c \in R \setminus \emptyset$ pro rovnici $b \cdot x = c$ existuje právě jedno řešení $x = \frac{c}{b}$

\subsubsection*{Poznámka}
Dvojici $(M, \circ), \textrm{kde} \quad \circ: M \times M \to M$,
$\circ$ je asociativní na M, neutrální prvek náleží M a pro každý prvek M inverzní prvek náleží M, říkáme grupa.

\subsection{Hierarchie}

\subsubsection*{Definice 2.1}
\textbf{Grupoid (magma)} je uspořádaná dvojice $(M, \circ)$,
kde $M$ je neprázdná množina a $\circ$ je binární operace na $M$.
\begin{itemize}
    \item \textbf{Pologrupa} (semigrupa) je grupoid $(M, \circ)$,
        kde operace $\circ$ je asociativní pro všechny prvky $M$.
    \item \textbf{Monoid} je pologrupa $(M, \circ)$,
        kde $\exists e, \forall a, \quad e,a \in M,$ \[a \circ e = e \circ a = a.\]
    \item \textbf{Grupa} je monoid $(M, \circ)$ a $e \in M$ je neutrální prvek,
        kde $\forall a \exists a^{-1}, \quad e,a \in M,$ \[a \circ a^{-1} = a^{-1} \circ a = e.\]
    \item \textbf{Komutativní (abelovská) grupa} je grupa $(M, \circ)$,
        kde operace $\circ$ je komutativní na $M$.
\end{itemize}

\subsubsection*{Poznámka}
O množině $M$ mluvíme také jako o \textit{nosiči} grupy $(M, \circ)$

\subsubsection*{Poznámka}
Binární operace je \textit{na} $M$ což znamená, že $\circ: M \times M \to M$,
také můžeme říci, že množina $M$ ja uzavřená vůči operaci $\circ$.

\subsection{Neutrální a inverzní prvky}

\subsubsection*{Věta 4.1}
V monoidu existuje právě jeden neutrální prvek.

\subsubsection*{Věta 4.2}
V grupě má každý prvek právě jeden inverzní prvek.

\subsection{Znázornění grup}

\subsubsection*{Poznámka}
Pokud má množina M konečný počet prvků, pak strukturu dvojic $(M, \circ)$,
lze kompletně zachytit Cayleyho tabulkou. Cayleyho tabulka pro $(M, \circ), |M|=n$
je tabulka $n \times n$, kde záhlaví sloupců i řádků jsou stejně seřazené prvky $M$,
políčko $p_{a,b}$ pro $a$-tý řádek a $b$-tý sloupec má hodnotu $a \circ b$.

\subsubsection*{Poznámka}
Latinský čtverec pro $n$ prvkovou množinu $M$ je tabulka $n \times n$,
kde v každém řádku a sloupci, je každý prvek $M$ právě jednou.

\subsubsection*{Věta 5.2}
Cayleyho tabulka každé grupy tvoří latinský čtverec.

\subsubsection*{Věta 5.3}
V každé grupě $(M, \circ)$ jde jednoznačně dělit.
 Tzn. $\forall a,b \in M$ mají rovnice \[a \circ x = b, \quad y \circ a = b\] jediné řešení.

\subsubsection*{Poznámka}
Grupu $(M, \circ)$ s konečným počtem prvků $M$ lze vizualizovat pomocí \textit{Cayleyho orientovaného grafu}.
Cayleyho orientovaný graf \[(V,E), \quad V=M, \quad E=\{(a,b): b=a \circ c, \quad \forall a \in M, \forall c \in N \subset V\}\]

\section{Přednáška 2.}

\subsection{Podgrupy}

\subsubsection*{Poznámka}
Hledáme-li podgrupu $(N, \circ)$ grupy $(M, \circ)$ tak aby obsahovala prvek $m$,
těleso musí zůstat grupou, proto musí také obsahovat všechny prvky tak, aby množina $N$ byla uzavřená na operaci $\circ$,
dále musí obsahovat neutrální prvek $e$, a inverzní prvek pro všechny prvky $N$.
Takovou podgrupu nazýváme \textbf{podgrupa generovaná množinou $\{m\}$}.

\subsubsection*{Definice 6.2 Podgrupa (subgroup) $(N, \circ)$}
Buď grupa $G = (M, \circ)$, podgrupa $H = (N, \circ)$ je libovolná dvojice, kde
\begin{itemize}
    \item $N \subset M$
    \item $(N, \circ)$ je grupa.
\end{itemize}

\subsubsection*{Poznámka}
Každá grupa $(M, \circ)$, kde $|M| \geq 2$ má vždy podrgrupy
\begin{itemize}
    \item $(\{e\}, \circ), e \in M$
    \item $(N, \circ), N = M$
\end{itemize}
těmto dvěma podgrupám říkáme \textbf{triviální podgrupy}. Ostatní podgrupy nazýváme \textbf{valstní (proper)}.

\subsubsection*{Věta 6.3}
Buď grupa $G = (M, \circ)$, pro každé $i$ z indexové množiny $I$ buď $H_{i}$ podgrupa $G$, pak 
\[H' = \bigcap_{i \in I} H_i\] je také podgrupa $G$.

\subsubsection*{Věta 6.4}
Buď grupa $G = (M, \circ), N \subset M \wedge N \neq \emptyset$,
pak libovolná dvojice $(N, \circ)$ je podgrupa právě tehdy když
\[ \forall a,b \in N, a \circ b^{-1} \in N \]

\subsection{Lagrangeova věta}

\subsubsection*{Definice 7.1 Řád (order)}
\textbf{Řádem grupy} $G = (M, \circ)$ nazýváme počet prvků $M$,
jeli počet prvků nekonečný, i řád je nekonečný, podle řádů rozdělujeme grupy na \textbf{konečné} a \textbf{nekonečné.}
Řád grupy $G$ značíme $\#G$ (nebo také $|G| = ord(G)$).

\subsubsection*{Věta 7.3 Lagrangeova}
Buď $H = (N, \circ)$ podgrupa konečné grupy $G = (M, \circ)$, potom řád $H$ dělí řád $G$.

\subsubsection*{Věta 7.4 Sylowova}
Buď grupa konečná $G$ řádu $n$ a $p$ prvočíselný dělitel $n$. Pokud $p^k$ dělí $n$ (pro $k$ přírozená),
potom existuje podgrupa $G$ řádu $p^k$. (Pro $k = 1$ též Cauchyho věta).

\subsection{Generující množiny a generátor grup}

\subsubsection*{Věta 8.1}
Buď grupa $G = (M, \circ)$ a $N \subset M \wedge N \neq \emptyset$, pak množina
\[ \langle N \rangle = \bigcap\{H: \textrm{$H$ je podgrupa grupy $G$ obsahující $N$}\} \]
spolu s operací $\circ$ tvoří podgrupu grupy $G$ obsahující prvek $N$.

\subsubsection*{Věta 8.2}
Podgrupu $\langle N \rangle$ grupy $G = (M, \circ), N \subset M \wedge N \neq \emptyset$
nazýváme \textbf{podgrupou generovanou množinou $N$}.
O množinu $N$ pak nazýváme jako \textbf{generující množinu} grupy $\langle N \rangle$.
V případě jednoprvkové generující množiny zavádíme značení $\langle a \rangle = \langle \{ a \} \rangle$
nazýváme jednoprvkovou množinu \textbf{generátor} grupy $\langle a \rangle$.

\subsubsection*{Poznámka}
Pro grupu $G = (M, \circ)$ s neutrálním prvkem $e \in M$
pro každý prvek $g \in M$ a $n \in \mathbb{N}$ zavádíme $n$-tou a $-n$-tou mocninu takto.

\begin{align*}
    &g^0 = e \\
    &g^1 = g \\
    &g^2 = g \circ g \\
    &g^n = g \circ g \circ g ... \circ g \quad \textrm{($n$-krát)} \\
    &g^{-2} = g^{-1} \circ g^{-1} \\
    &g^{-n} = (g^{-1})^{n}
\end{align*}

\subsubsection*{Věta 8.5}
Buď grupa $G = (M, \circ)$ a podmnožina $N \subset M \wedge N \neq \emptyset$,
potom všechny prvky grupy $\langle N \rangle$ lze získat pomocí grupového obalu
\[\langle N \rangle = \{a_{1}^{k_1} \circ a_{2}^{k_2} \circ ... a_{n}^{k_n} \circ: n \in \mathbb{N}, k_i \in \mathbb{Z}, a_i \in N\} \]

\subsubsection*{Důsledek}
$ \langle N \rangle = \{a^k: k \in \mathbb{Z}\}$

\subsection{Cyklické grupy}

\subsubsection*{Věta 9.1}
Grupa $\mathbb{Z}^+_n$ je rovna $\langle k \rangle, k \in \mathbb{Z}^+_n$
tehdy a jen tehdy, když $k$ a $n$ jsou nesoudělná čísla.

\subsubsection*{Definice 9.4 Cyklická grupa (cyclic group)}
Grupa $G = (M, \circ)$ se nazývá \textbf{cyklická},
pokud existuje $a \in M, \langle a \rangle = G$. Prvek $a$ se nazývá \textbf{generátor} cyklické grupy $G$.

\subsubsection*{Definice 9.5}
Buď $g$ prvek grupy $G = (M, \circ)$, existuje-li $m \in \mathbb{N}$ takové, že $g^m = e$,
pak nejmenší takové $m$ nazýváme \textbf{řádem prvku g}. Neexistuje-li takové $m$, pak prvek $g$ má řád nekonečno.
Řád prvku $g$ značíme $ord(g)$.

\subsubsection*{Poznámka}
Řád prvku $g$ se rovná řádu množiny generované $g$, platí tedy rovnost
\[ord(g) = \#\langle g \rangle\]
Dále platí, že $ g^k = e \Leftrightarrow k = l \cdot ord(g), l \in \mathbb{Z}$

\subsubsection*{Věta 9.6}
Grupa $\mathbb{Z}^{\times}_{n}$ je cyklická právě tehdy když $n \in \{2, 4, p^k, 2p^k\}, k \in \mathbb{N}$
a $p$ je liché prvočíslo.

\subsubsection*{Poznámka}
Obecně najít generátor grupy není jednoduché, (třeba pro grupy $\mathbb{Z}^{\times}_{n}$).
Pokud však jeden známe, je jednoduché najít všechny ostatní.

\subsubsection*{Věta 9.7}
Je-li $G = (M, \circ)$ cyklická grupa řádu $n$ a $a$ nějaký její generátor.
Potom $a^k$ je také její tehdy a jen tehdy, když $n$ a $k$ jsou nesoudělné, tedy $gcd(n,k) = 1$

\subsubsection*{Důkaz}
\textbf{!!! brutus !!!}

\subsubsection*{Poznámka}
$\varphi(n)$ je \textbf{Eulerova funkce},
 každému $n \in \mathbb{N}$ přiřazuje počet přirozených čísel z rozmezí $\langle 1;n)$, která jsou s ním nesoudělná.

\subsubsection*{Věta 9.8}
V cyklické grupě řádu $n$ je počet generátorů roven $\varphi(n)$.

\subsubsection*{Věta 9.9}
Libovolná podgrupa cyklické grupy je opět cyklická grupa.

\subsection{(Malá) Fermatova věta}

\subsubsection*{Věta 10.1}
V grupě $G = (M, \circ)$ řádu $n$ pro libovolný prvek $a \in M$ platí
$a^n = e$, kde $e$ je neutrální prvek. 

\subsubsection*{Poznámka}
Grupa $ \mathbb{Z}^{\times}_{p}$ je cyklická a řádu $p - 1$.

\subsubsection*{Věta 10.2}
Pro libovolné $p$ a libovolné $1 < a < p$
\[a^{p-1} \equiv 1 (mod \quad p). \quad (a^n \equiv a (mod \quad p))\] 

\section{Přednáška 3.}

\subsection{Homomorfismy a izomorfismy}

\subsubsection*{Definice 11.1}
Buď $G = (M, \circ_G), H = (N, \circ_H)$ dva grupoidy,
zobrazení $h: M \to N$, nazveme \textbf{homomorfismem G do H}, jestliže
\[ \forall a,b \in M, h(a \circ_G b) = h(a) \circ_H h(b)\]
Je-li navíc $h$ injektivní, resp. surjektivní, resp. bijektivní, říkáme,
že jde o \textbf{monomorfismus}, resp. \textbf{epimorfismus}, resp. \textbf{izomorfismus}.

\subsubsection*{Definice 11.2}
Grupy $G = (M, \circ_G), H = (N, \circ_H)$ nazýváme \textbf{izomorfní},
 právě tehdy když existuje izomorfismus $h: M \to N$, také říkáme, že G je izomorfní s H.

\subsubsection*{Poznámka}
Vlastnost dvou grup být izomorfní je relace ekvivalence na množině všech grup.

\subsubsection*{Věta 11.3}
Buď homomorfismus grupy $G = (M, \circ_G)$ do grupoidu $H = (N, \circ_H)$ $h: M \to N$,
potom $h(G) = (h(M), \circ_H)$ je grupa.

\textbf{Důsledky}: Je-li $H$ grupa, potom
\begin{itemize}
    \item neutrální prvek G se zobrazí na neutrální prvek H
    \item inverzní prvek se zobrazí na inverzní prvek H $h(x^{-1}) = h(x)^{-1}$
    \item je-li $h$ homomorfismus z $G \to H$, $h(G)$ je podgrupa $H$
\end{itemize}

\subsubsection*{Věta 11.4}
Libovolné dvě nekonečné cyklické grupy jsou izomorfní.
Pro každé $n \in \mathbb{N}$ jsou libovolné dvě cyklické grupy řádu $n$ izomorfní.

\subsubsection*{Poznámka}
$(\mathbb{Z},+), (\mathbb{Z},+_n)$ jsou jediné cyklické grupy až na izomorfismus.

\subsubsection*{Poznámka Kleinova grupa}
$(\mathbb{Z}_2 \times \mathbb{Z}_2, \circ)$, kde
\[\mathbb{Z}_2 \times \mathbb{Z}_2 = \{(0,0), (0,1), (1,0), (1,1)\}\]
$\circ$ je modulo $2$ po složkách $(1,0) \circ (1,1) = (0,1)$
Kleinova grupa není cyklická, nemůže být tedy izomorfní s $\mathbb{Z}^+_4$

\subsubsection*{Věta 11.6}
Existují pouze dvě neizomorfní grupy řádu 4.

\subsubsection*{Poznámka (Symetrická grupa)}
Symetrickou grupou množiny $\{1,2,...,n\}$
nazveme množinu všech permutací s operací skládání zobrazení a značíme ji $S_n$.

\subsubsection*{Poznámka}
Permutace můžeme zadat výčtem hodnot
\[
\begin{pmatrix}
    1 & 2 & ... & n\\
    \pi(1) & \pi(2) & ... & \pi(n)
\end{pmatrix}
\]
první řádek můžeme navíc vynechat.
\textbf{Skládání permutací} 
\[
    \begin{pmatrix} 1 & 2 & 4 & 3 & 5 \end{pmatrix} \circ
    \begin{pmatrix} 2 & 1 & 3 & 5 & 4 \end{pmatrix} =
    \begin{pmatrix} 2 & 1 & 4 & 5 & 3 \end{pmatrix}
\]
Skládání zobrazení je asociativní, matice \[ \begin{pmatrix} 1 & 2 & ... & n \end{pmatrix} \] je neutrální prvek
 a inverzní prvek je inverzní zobrazení, $S_n$ je tedy opravdu grupa a má řád $n!
 $

\subsubsection*{Poznámka}
Podgrupy symetrické grupy $S_n$ nazýváme \textbf{grupami permutací}.

\subsubsection*{Věta 11.8 (Cayleyova)}
Každá konečná grupa $G$ je izomorfní s nějakou grupou permutací. 

\subsection{Aplikace teorie grup v kryptografii}

\subsubsection*{Definice 12.1 (Problém diskrétního logaritmu na grupě $G = (M, \cdot)$)}
Buď $G = (M, \cdot)$ cyklická grupa řádu $n$, $\alpha$ její generátor a $\beta$ její prvek.
Řešit problém diskrétního logaritmu znamená najít číslo $1 \leq k \leq n$, takové, že
\[ \alpha^k = \beta \].

\subsubsection*{Definice 12.2 (Problém diskrétního logaritmu na grupě $G = (M, +)$)}
Buď $G = (M, +)$ cyklická grupa řádu $n$, $\alpha$ nějaký její generátor a $\beta$ její prvek.
Řešit problém diskrétního logaritmu znamená najít číslo $1 \leq k \leq n$, takové, že
\[ k \times \alpha = \beta \].

\subsubsection*{Poznámka}
Zahodíme li požadavek na cykličnost grupy $G$, problém má řešení pouze pokud,
prvek $\beta$ patří do cyklické podgrupy $\langle \alpha \rangle$ generované $\alpha$.

\subsubsection*{Příklad, kde to lze snadno}
Mějme grupu $\mathbb{Z}^+_p$, generátor $\alpha$, prvek $\beta$, řešíme
\[k \times \alpha = \beta \mod p\]
Pomůžeme si grupou $\mathbb{Z}^\times_p$, kde řešením bude,
\[k = \beta \cdot \alpha^{-1} \mod p\]

\subsubsection*{Obecně}
Není znám rozumně rychlý algoritmus řešící problém diskrétního logaritmu.

V případě grupy $\mathbb{Z}^\times_p$ znám algoritmus úměrný $\sqrt{p}$.
Inverzní operaci, tedy mocnění umíme naopak velmi rychle (metoda opakovaných čtverců).

Dostáváme tedy \textbf{jednosměrnou (one-way) funkci}, kterou lze použít pro kryptografii.
Spočítat $\beta = \alpha^x \mod p$ je snadné, známe-li $x, \alpha, p$,
naopak spočítat $x$ známe-li $\alpha, \beta, p$ je velmi obtížné.

\textbf{Poznámka}: Pro konstrukci RSA šifry bylo použita jednosměrná funkce násobení prvočísel.
Vynásobit dvě velká prvočísla je snadný úkol. Rozložit velké číslo na dvě prvočísla je velmi obtížné.

\subsubsection*{Diffie-Hellman Key Exchange}
\begin{enumerate}
    \item Inicializace:
    \subitem Alice zveřejní generátor $\alpha$ a prvočíslo $p$.
    \subitem Alice si zvolí $x$ a spočítá svůj veřejný klíč $A = \alpha^x \mod p$.
    \subitem Bob si zvolí tajný klíč $y$ a spočítá svůj veřejný klíč $B = \alpha^y \mod p$.
    \subitem Výměna klíčů.
    \item Komunikace:
    \subitem Alice počítá zprávu jako $k_{AB} = B^x \mod p$.
    \subitem Bob počítá zprávu jako $k_{AB} = A^y \mod p$.
\end{enumerate}

\textbf{Fakta}
\begin{itemize}
    \item Operace mocnění je v $\mathbb{Z}^\times_p$ komutativní
     a tedy vypočtené $k_{AB}$ je pro oba stejné, protože $k_{AB} \equiv \left( \alpha^{y} \right)^x \equiv \left( \alpha^{x} \right)^y \mod p$
    \item Mocnění není výpočetně náročné (square nad multiply).
    \item Řešení diskrétního logaritmu je velmi náročné.
\end{itemize}


\section{Přednáška 4.}

\subsection{Množiny se dvěma binárními operacemi}

\subsection{Okruhy polynomů}

\subsection{Konečná tělesa}

\subsection{Aplikace konečných těles v kryptografii}

\section{Přednáška 5.}

\end{document}
